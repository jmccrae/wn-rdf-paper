\documentclass[10pt, a4paper]{article}
\usepackage{lrec2006}
\usepackage{graphicx}

\title{Publishing and Linking WordNet using lemon and RDF}

\name{John P. M\textsuperscript{c}Crae\textsuperscript{*}, Christiane Fellbaum\textsuperscript{$\dagger$} and Philipp Cimiano\textsuperscript{*}}

\address{ \textsuperscript{*}University Bielefeld, \textsuperscript{$\dagger$}Princeton University \\
               CITEC Building, Inspiration 1, 33615 Bielefeld; 35 Olden Street, Princeton \\
               \{jmccrae, cimiano\}@cit-ec.uni-bielefeld.de, fellbaum@princeton.edu \\}


\abstract{Each article must include an abstract of 150 to 200 words in Times 9 pt with interlinear spacing of 10 pt.
 The heading Abstract should be centred, font Times 10 bold. This short abstract will also be used for printing a Booklet of Abstracts 
containing the abstracts of all papers presented at the Conference. \\ \newline \Keywords{keyword A, keyword B, keyword C}}



\begin{document}

\maketitleabstract

\section{Introduction}

WordNet is one the first and still most widely used resources for natural
language processing. In the time since the first version of WordNet was released many
resources have been produced that represent complementary information\cite{?,?} or extend
the WordNet model to new languages\cite{?,?}. In contrast, in recent years we have seen
the development of Web Technologies for the representation of language resources
and in particular, the use of linked data. This has lead to a linguistic linked
open data cloud, which is constructed by linking resources and publishing them
on the web using RDF. Linked data, as proposed by Berners-Lee\cite{?}, has four
main principles for publishing data: firstly, the use of URIs to identify
objects; secondly, that these URIs should be resolvable; thirdly, that semantic
information is returned, using standards such as RDF and finally, that links are
provided to other resources. Chiarcos~\emph{et al.}\cite{?} discuss the
application of this to linguistic data and show that this model has notable
advantages over standard approaches to data modelling, in particular they
outline the following:

\begin{enumerate}
    \item Representation: Graph-based models are a method that can represent any
        form of language resource.
    \item Structural interoperability: By using RDF graphs and URIs datasets can
        be merged with no effort.
    \item Federation: Multiple datasets can easily be drawn from different
        sources in the web and used together seamlessly.
    \item Conceptual interoperability: Linking to common data category
        repositories allows common definitions to be inferred.
    \item Ecosystem: Building on standards such as RDF, allows the use of common
        tools, including databases.
    \item Dynamic import: Data on the web is not static and as such errors can
        be corrected after publication.
    \item Expressivity: The use of other Semantic Web models allows the easy
        expression of metadata, provenance and ontological constraints on the
        data
\end{enumerate}

In this paper we describe our experience
in publishing WordNet following the linked data principles. While this is not
the first version of WordNet to be published as linked data~\cite{?, ?, ?}, this
version has several advantages, firstly that it is well-linked to many
resources, secondly it uses an open model in \emph{lemon} and finally, that as
it is integrated with the development of WordNet, and as such will be updated alongside
future releases of WordNet.

\section{Background}

\subsection{WordNet}

\subsection{lemon}

\subsection{Linguistic Linked Data}

\section{Representing WordNets with lemon}

\section{Linking WordNet}

\section{Related Work}

\section{Conclusion}

\bibliographystyle{lrec2006}
\bibliography{rdf-wordnet}

\end{document}

