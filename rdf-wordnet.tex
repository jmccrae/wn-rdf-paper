\documentclass[10pt, a4paper]{article}
\usepackage{lrec2006}
\usepackage{graphicx}
\usepackage{todonotes}

\title{Publishing and Linking WordNet using lemon and RDF}

\name{John P. M\textsuperscript{c}Crae\textsuperscript{*}, Christiane Fellbaum\textsuperscript{$\dagger$} and Philipp Cimiano\textsuperscript{*}}

\address{ \textsuperscript{*}University Bielefeld, \textsuperscript{$\dagger$}Princeton University \\
               CITEC Building, Inspiration 1, 33615 Bielefeld; 35 Olden Street, Princeton \\
               \{jmccrae, cimiano\}@cit-ec.uni-bielefeld.de, fellbaum@princeton.edu \\}


\abstract{Each article must include an abstract of 150 to 200 words in Times 9 pt with interlinear spacing of 10 pt.
 The heading Abstract should be centred, font Times 10 bold. This short abstract will also be used for printing a Booklet of Abstracts 
containing the abstracts of all papers presented at the Conference. \\ \newline \Keywords{keyword A, keyword B, keyword C}}

\newcommand{\lemon}[0]{\emph{lemon}}

\begin{document}

\maketitleabstract

\section{Introduction}

WordNet is one the first and still most widely used resources for natural
language processing. In the time since the first version of WordNet was released many
resources have been produced that represent complementary information\cite{?,?} or extend
the WordNet model to new languages\cite{?,?}. In contrast, in recent years we have seen
the development of Web Technologies for the representation of language resources
and in particular, the use of linked data. This has lead to a linguistic linked
open data cloud, which is constructed by linking resources and publishing them
on the web using RDF. Linked data, as proposed by Berners-Lee\cite{?}, has four
main principles for publishing data: firstly, the use of URIs to identify
objects; secondly, that these URIs should be resolvable; thirdly, that semantic
information is returned, using standards such as RDF and finally, that links are
provided to other resources. Chiarcos~\emph{et al.}\cite{?} discuss the
application of this to linguistic data and show that this model has notable
advantages over standard approaches to data modelling, in particular they
outline the following:

\begin{enumerate}
    \item Representation: Graph-based models are a method that can represent any
        form of language resource.
    \item Structural interoperability: By using RDF graphs and URIs datasets can
        be merged with no effort.
    \item Federation: Multiple datasets can easily be drawn from different
        sources in the web and used together seamlessly.
    \item Conceptual interoperability: Linking to common data category
        repositories allows common definitions to be inferred.
    \item Ecosystem: Building on standards such as RDF, allows the use of common
        tools, including databases.
    \item Dynamic import: Data on the web is not static and as such errors can
        be corrected after publication.
    \item Expressivity: The use of other Semantic Web models allows the easy
        expression of metadata, provenance and ontological constraints on the
        data
\end{enumerate}

In this paper we describe our experience
in publishing WordNet following the linked data principles. While this is not
the first version of WordNet to be published as linked data~\cite{?, ?, ?}, this
version has several advantages, firstly that it is well-linked to many
resources, secondly it uses an open model in \emph{lemon} and finally, that as
it is integrated with the development of WordNet, and as such will be updated alongside
future releases of WordNet.

\section{Background}

\subsection{WordNet}

\todo{Introduce description of WordNet}

\subsection{lemon}

\lemon{} is a model that has been proposed\cite{} for the representation of
lexicons relative to ontologies. As such, this model is well suited to the
representation of semantic networks such as WordNet and defines many useful
features for linking a WordNet to wider objects in the Semantic Web. \lemon{}
models lexicons by means of a core consisting of the following elements:

\begin{itemize}
  \item A \emph{lexical entry} which represents a single word or multi-word
    unit.
  \item A \emph{lexical sense}, representing a meaning of that word, which
    contains a \emph{reference} to a concept in the ontology.
  \item \emph{Forms}, which are inflected version of the entry, and associated
    with a string \emph{representation}.
\end{itemize}

\subsection{Linguistic Linked Data}

The application of linked data technology to the representation of linguistics
resources has been spearheaded by the OKFN Working Group on Linguistics\cite{},
and they have been mapping the progress of this project by means of a cloud
diagram showing how all the existing resources are linked (figure
\ref{llod-cloud}). As outlined above, there are many key advantages to the use
of this technology for language resources, however the cloud has until now
lacked a central node. As WordNet is the most widely referenced language
resource we believe that WordNet can act as a nucleus for linguistic linked data
in the way that DBpedia has for the wider cloud.

\section{Representing WordNets with lemon}

\begin{figure}
  \missingfigure{Example of entry graph}
\end{figure}

\begin{table}
  \begin{tabular}{p{50mm}|c}
     & Number of triples \\
    \hline
    Links to VerbNet & ? \\
    Links to DBpedia & ? \\
    Links to LexVo & ? \\
    Links to lemonUBY & ? \\
    Other & ? \\
    \hline
    Total & ? \\
  \end{tabular}
\end{table}


As \lemon{} is a model for representing lexica relative to ontologies it is not
obviously clear how it can be used to represent a WordNet. It is clear that the
words of WordNet can be called lexical entries and the word senses correspond
well to the concept of lexical senses. WordNet has lemmas and a separate list of
variants of these, and as such we create a canonical form for each lemma and an
other form for each of these variants. As there is currently no indication in
WordNet of what grammatical properties these variants have we do not distinguish
these forms by means of annotation. As a \lemon{} is a model for
ontology-lexica, the main question is what the reference of the sense should be.
We approach this by saying that the synsets of WordNet are the ontological
references, but instead of assigning them a formal ontological type (e.g.,
class, property or indiviual) we instead use the SKOS\cite{} vocabulary and type
them as concepts. This allows us to capture the nature of synsets without
ontologizing the semantic network as in \cite{}. 

The other key question is the identifiers we use for each element in the data.
We do not follow previous exports such as \cite{} in assigning new identifiers
but instead attempt to use the existing identifiers in WordNet. Furthermore, as
WordNet has released several versions and is still under development, we feel it
is important to include the version number in the URI. As such, we use the
following scheme for URIs:

\begin{itemize}
  \item Each lexical entry is represented by means of the URL-encoded lemma and
    then a dash followed by the part of speech as a single letter (i.e., `n(oun)',
    `v(erb)', `a(djective)', `r (adverb)', `s(entence)' or `p(article)').
  \item Senses and forms in the model use the entry URI and add a fragment
    identifier. For forms, as there is no previous identifier in WordNet we
    simply use {\tt CanonicalForm} and {\tt Form-n} where {\tt n} is a number.
    For senses, the fragment is simply the sense identifier from WordNet
  \item Synsets are similarly the 8 figure `offset' code from the WordNet
    database, followed by a dash and the part of speech as a single letter.
\end{itemize}

As such an example of the URI scheme is shown in figure \ref{uri-examples}.

\begin{figure*}
\begin{verbatim}
http://wordnet-rdf.princeton.edu/wn31/cat-n
http://wordnet-rdf.princeton.edu/wn31/cat-n#CanonicalForm
http://wordnet-rdf.princeton.edu/wn31/cat-n#???
http://wordnet-rdf.princeton.edu/wn31/00001740-a
\end{verbatim}
%\label{URI scheme of RDF WordNet\label{uri-examples}}
\end{figure*}

A Python framework based on RDFlib\cite{} is used to serve the website and
provide SPARQL access to the data.

\section{Linking WordNet}

In addition to providing a RDF version of WordNet we also incorporated a number
of extra resources founded from other sources into the RDF data. In particular
we include the following elements

\begin{itemize}
  \item For verbs, where extant we include mappings to VerbNet\cite{}. As
    VerbNet does not currently have a linked data version, we simply link to the
    PHP page of the web site.
  \item We include translations from the Open Multilingual WordNet\cite{}
    collection as simple labels on the synsets, identified by the use of
    language codes.
  \item We have mapped previous mappings to LexVo\cite{} and DBpedia\cite{} to
    use the current identifiers in WordNet.
  \item We include links to the W3C WordNet 2.0 export\cite{}.
  \item We have created new links to the lemonUby\cite{} resource.
\end{itemize}

In addition to these links we also provide support for legacy resources by
adding URL mappings from previous versions of WordNet identifiers to the most
recent version. These mappings are based on the mappings of \cite{}

\section{Related Work}

This work does not represent the first version of WordNet made available in RDF,
in fact \cite{van2006conversion,mccrae2012integrating} have made previous
version available directly in WordNet. Furthermore, WordNet has been
incorporated into various larger resources including
BabelNet~\cite{navigli2010babelnet,ehrmann2014},
UBY~\cite{gurevych2012uby,eckle2014lemonuby}. These projects however have mostly
been fixed to using a single version of WordNet. In constrast, we view our work
as more related to task of providing universal identifiers for words as in the
work of the Global WordNet Grid~\cite{pease2008building}.  

\section{Conclusion}

In this paper we have presented the creation of a WordNet RDF version that is
linked with the development of the existing model as well as incorporates a
large number of links to other resources on the web. As such we believe that
this node will constitute a key central node for the expansion of the linguistic
linked open data cloud

\bibliographystyle{lrec2006}
\bibliography{rdf-wordnet}

\end{document}
